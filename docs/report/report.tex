\documentclass{article}
\usepackage[margin=2cm]{geometry}
\usepackage{graphicx}
\usepackage{hyperref}
\hypersetup{
    colorlinks=true,
    linkcolor=blue,
    filecolor=magenta,      
    urlcolor=cyan,
}

\begin{document}
\title{{\Huge Net Computing} \\[.5cm] {\Large Nodes Measurement System}}
\author{
\begin{tabular}{r l}
	Martijn Luinstra & s\,2199289 \\
	Emilio Oldenziel & s\,2509679 \\
	Yannick Stoffers & s\,2372061
\end{tabular}
}

\maketitle

\section{Introduction}
    % What our application does %
    We created an nodes measurement system where a user can monitor nodes 
    (or any other machine) in a network. To do this we created an muliplatform
    application existing of 3 parts, the first one is the server which is the
    part which is the core of our application, this server handles the setup of
    the nodes and clients, handles messages comming from both and does the 
    needed administration. The second part is the node part, this piece of 
    software is run on any machine and connects to the central server, then it
    measures the machines hardware and sends this to the server. The last part 
    are the clients which can connect to the central server with their browser 
    to see a dashboard with the measurements from all nodes.
    

\newpage
\section{Architecture}
    % Overview of how the system is set-up %
    The system consists of a central server, nodes that are measured and clients
    that are connected to the central server with their browser. The nodes send 
    their measurements to the central server using message queueing, the central
    server has a worker that handles the message queue and takes the measurement
    from the queue and than processes them. The worker saves the measurement in 
    a database for archiving purposes and sends the measurement directly to the 
    client's browser using the browser's websocket. More nodes that have to be
    measured can be added by using web services, the node asks the central 
    server if he can be monitored, if that is possible, the central server gives
    setup data to the node like what its id is and where he has to queue his 
    measurements.
    
    \includegraphics[scale=0.45]{architecture.png}

\section{Design choices}
    %explain why we used one central server with workers instead of p2p
    We chose to have a centralised server so that we have one point that 
    does the 'bookkeeping' of which nodes are monitored, but also to 
    make it simpler that nodes and client connect to the same point.
    
    %Why we chose python
    The language that we used for this project is Python, we chose that 
    because we all a lot more experience with Python than with Java. For 
    projects like this where we have to show that we understand the concepts 
    and techniques Python is nicer because we don't have to deal with the
    generics of Java and can concentrate more on the main concepts. Speed is 
    not an issue in this system because Python is fast enough to handle all 
    the operations without human noticeable latency.
    
    % explain why we used JSON %
    We used JSON as datatype for our communication, mainly because it is very 
    simple to use in python. There is a standard JSON library in python 
    that can create a JSON object from a dictionary (python version of a 
    HashMap) and vice versa.
    In this way we could serialize our data in a simple way and then send it.
    JSON is nowadays the most used data-type for sending objects over the 
    internet because of its simplicity and human-readability. %vague remark%
    
\section{Application Flow}
    First the central server and its workers are started. A node can ask the 
    server if he can be observed and gives his name and ip-address, the server 
    responds with No or gives the setup data. If the node received the setup 
    data the node will start a worker thread that does measurements from the 
    hardware and queues this to the message queue of the central worker, the 
    node will continue doing this every second till the thread is stopped. 
    The central server's message queue worker will save the measurement and 
    send it directly to all connected browsers. If a node has a major difference
    in some metric (e.g. the CPU temperature will increase very rapidly) this 
    will be seen on the dashboard on the browser. % How to do PYRO Stuff %
    

\section{Techniques}
    %Flask (API, Server, Models, sockets), RabbitMQ(Pika), PYRO
    We made use of multiple frameworks for this project. For the central server
    we used \href{http://flask.pocoo.org/}{Flask} with the 
    \href{https://flask-restful.readthedocs.io/en/0.3.5/}{Flask-Restful} API to
    do the initial setup of a node, this API uses REST to handle requests. As
    server for the API we used the development server of Flask, this is a 
    lightweight server that has everything to demonstrate the application. To 
    store the measurements 
    \href{http://flask-sqlalchemy.pocoo.org/2.1/}{Flask SQL-alchemy} is used 
    with an SQLite database for simple storage of data from the demo.
    For Message queueing RabbitMQ is used, to use RabbitMQ with python we 
    used \href{https://github.com/pika/pika}{pika} which is a Python interface
    to RabbitMQ. 

\section{Difficulties}
    % RabbitMQ? %

\section{Conclusion}
    In this project we wanted to build something that was simple but had all
    compulsary components in it. For each part we picked the technique that 
    was the most logical, mostly based on frequecy of use and speed, but also in
    such a way that all parts where implemented with a different one. Using 
    Python we had a lot of libraries and frameworks to choose from, we used 
    the ones that we though that got the job done in the simples way or we 
    already had experience with such as Flask. Doing this it saved us a lot of 
    time because we already know and/or minimized the stuggles that come with 
    such frameworks. 

\section{Evaluation}
    All in all we think that we have learned a lot of things that are very 
    practical to have knowledge of. This can be become very handy for later 
    projects that need a way of communication between machines. We now know 
    which techinques are most suitable for certain applications or architectures.
    % more bladibladi bla ...

\end{document}